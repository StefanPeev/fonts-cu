\documentclass[11pt]{ltxdoc}
\usepackage[usenames,dvipsnames,svgnames,table]{xcolor}
\usepackage{fontspec}
\usepackage{xltxtra}
% code borrowed from Polyglossia documentation — Thanks!
\definecolor{myblue}{rgb}{0.02,0.04,0.48}
\definecolor{lightblue}{rgb}{0.61,.8,.8}
\definecolor{myred}{rgb}{0.65,0.04,0.07}
\usepackage[
    bookmarks=true,
    colorlinks=true,
    linkcolor=myblue,
    urlcolor=myblue,
    citecolor=myblue,
    hyperindex=false,
    hyperfootnotes=false,
    pdftitle={Church Slavonic fonts},
    pdfauthor={Aleksandr Andreev},
    pdfkeywords={Church Slavic, Church Slavonic, Old Church Slavonic, Old Slavonic, fonts, Unicode}
    ]{hyperref}
\usepackage{polyglossia}
\setmainlanguage[variant=american]{english}
\setotherlanguages{russian,churchslavonic}

%% KEYBOARD DRIVER VERSION AND RELEASE DATES
\def\filedate{April 26, 2016}
\def\fileversion{version 0.1 $\beta$}

%% fontspec declarations:
\setmainfont{Linux Libertine O}
\setsansfont{DejaVu Sans}
\setmonofont[Scale=MatchLowercase]{DejaVu Sans Mono}
\newfontfamily{\slv}[Scale=MatchLowercase]{Ponomar Unicode}
\newfontfamily{\ust}[Scale=MatchLowercase]{Menaion Unicode}

\linespread{1.05}
\frenchspacing
\EnableCrossrefs
\CodelineIndex
\RecordChanges
% COMMENT THE NEXT LINE TO INCLUDE THE CODE
\AtBeginDocument{\OnlyDescription}

% FOR DRAWING KEY CAPS
\begin{document}

\title{Church Slavonic Fonts}
\author{Aleksandr Andreev \and Yuri Shardt \and Nikita Simmons\thanks{Comments may be directed to \href{mailto:aleksandr.andreev@gmail.com}{aleksandr.andreev@gmail.com}.}}
\date{\filedate \qquad \fileversion\\
\footnotesize (\textsc{pdf} file generated on \today)}

\maketitle
\tableofcontents

\section{Introduction}

Church Slavonic (also called Church Slavic, Old Church Slavonic
or Old Slavonic; ISO 639-2 code |cu|) is a literary language used by
the Slavic peoples; presently it is used as a liturgical language by the
Russian Orthodox Church, other local Orthodox Churches, as well
as various Byzantine-Rite Catholic and Old Ritualist communities.
The package \texttt{fonts-churchslavonic} provides fonts for
representing Church Slavonic text.

The fonts are designed to work with Unicode text encoded in UTF-8.
Texts encoded in legacy codepages (such as HIP and UCS) may be
converted to Unicode using a separate bundle of utilties.
See the \href{http://sci.ponomar.net/}
{Slavonic Computing Initiative website} for more information.

\section{License}

The fonts distributed in this package are dual-licensed under the GNU General Public License 
(version 3 or later) and the SIL Open Font License (version 1.1 or later). The SIL
Open Font License is preferred, since this is a FLOSS license intended for fonts.
Dual licensing under GNU GPL is maintained to allow embedding of these
fonts into GPL-licensed applications and for compatibility with other projects.

The fancy legal text:

The fonts distributed in this package are free software: you can redistribute them and/or modify
them, in whole or in part, EITHER under the terms of the GNU General Public License as published by
the Free Software Foundation, either version 3 of the License, or
(at your option) any later version OR under the terms of the SIL Open Font License,
version 1.1, or (at your option) any later version, without reserved font names.

As a special exception, if you create a document which uses any of these fonts, and embed the font or unaltered portions of the font
into the document, the font does not by itself cause the resulting document to be covered by the GNU General Public License. 
This exception does not however invalidate any other reasons why the document might be covered by the GNU General Public License. 
If you modify any of these fonts, you may extend this exception to your version of the fonts, but you are not obligated to do so. 
If you do not wish to do so, delete this exception statement from your version.

As free software, these fonts are distributed in the hope that they will be useful,
but WITHOUT ANY WARRANTY; without even the implied warranty of
MERCHANTABILITY or FITNESS FOR A PARTICULAR PURPOSE.  See the
GNU General Public License or the SIL Open Font License for more details.

\section{Introduction}

The package provides several fonts that are intended for working with Church Slavonic text
of various recensions and other texts related to Church Slavonic:
modern Church Slavonic text (“Synodal Slavonic”), historical printed Church Slavonic text
and manuscript uncial (ustav) Church Slavonic text (in either Cyrillic or Glagolitic),
as well as text in Sakha (Yakut), Aleut (Fox Island dialect), and Romanian (Moldovan)
Cyrillic, all written in the ecclesiastical script. The coverage of the various fonts agrees
with the guidelines for font coverage specified in \hyperref{http://www.unicode.org/notes/tn41/}
{Unicode Technical Note \#41: Church Slavonic Typography in Unicode}.
Generally speaking, it includes most (but not all) characters in the Cyrillic,
Cyrillic Supplement, Cyrillic Extended-A, Cyrillic Extended-B, Cyrillic Extended-C
(as of Unicode 9.0), Glagolitic, and Glagolitic Extension blocks of Unicode.
Characters not used in Church Slavonic, however, are not included (except for some
characers used in modern Russian, Ukrainian, Belarussian, Serbian and
Macedonian for purposes of compatibility with some applications.

\section{Installation and Usage}

If you are reading this document, then you probably have already downloaded
the font package. You may check if you have the most recent version by visiting
the \href{http://sci.ponomar.net/}{Slavonic Computing Initiative website}.

\subsection{Font Formats}

All fonts are currently available in two formats:

\begin{description}
% [\XeTeXpicfile "truetype.png"]
\item TrueType fonts, or, more precisely,
\hyperlink{OT}{OpenType} fonts with TrueType outlines;

\item \hyperlink{OT}{OpenType} fonts with
PostScript outlines (also called OpenType-CFF fonts).
%[\XeTeXpicfile "opentype.png"]
\end{description}

Note that fonts in these two formats have different file extensions:
\texttt{*.ttf} for TrueType, \texttt{*.otf} for OpenType-CFF fonts.
Both the TrueType version and the OpenType-CFF version support
the same set of advanced \hyperlink{OT}{OpenType} features.

The OpenType-CFF fonts use PostScript oputlines, based
on third-order (cubic) Bézier curves, while the TrueType fonts 
use second-order (quadratic) curves. There is also a significant difference in
hinting (grid fitting): TrueType instructions theoretically allow to
achieve much better quality of screen rendering than PostScript hinting.
However, since quality hinting is a very difficult and time-consuming process,
both the PostScript hinting and TrueType instructing of the fonts 
has been done automatically, so high quality grid fitting is not available.

Note that it is possible to install both the TrueType and OpenType-CFF versions
simultaneously. For this purpose, the TrueType fonts contain a “TT” suffix in their
font name/family name fields. Since all of the fonts have been drawn in cubic splines
(and then converted to quadratic for the TTF version), and since the TrueType instructions
have been automatically generated, the OpenType-CFF format may
theoretically give you better screen rendering quality, though in most situations
this will not be noticeable.

Note that only the TTF version supports \hyperlink{Graphite}{SIL Graphite},
so you will need to use the TrueType fonts if Graphite support is desired.
The choice of font format is thus largely a matter of taste on modern systems;
however the following considerations are in order for some environments:

\begin{itemize}

\item In older versions of OpenOffice.org, OpenType-CFF fonts 
were not properly embedded into PDF files. Moreover, under Unix-based
systems, OpenOffice.org could not access such fonts at all, so using TTF
versions was the only option. This was fixed in OpenOffice.org 3.2 and LibreOffice.

\item OpenOffice.org and LibreOffice, however, still have no mechanism to 
turn off and on advanced OpenType features, so if you plan to use optional typographic
features, you will need to use SIL Graphite, which is only available in the TTF version.

\item OpenType-CFF fonts are poorly supported in the Sun Java Development Kit,
so we recommend use of the TTF versions in Java programming situations and
on Android devices.

\item On Microsoft Windows, OpenType glyph positioning is not supported for glyphs
in the Private Use Area or characters outside of the Unicode 7.0 range. You will 
need to rely on \hyperlink{Graphite}{SIL Graphite} (only available in the TTF versions)
for positioning of combining Glagolitic characters and various glyphs in the PUA.

\end{itemize}

Note that Microsoft Windows checks the presence of
a digital signature in a TrueType font, considering this would allow to distinguish
“old” TrueType fonts from “modern” OpenType fonts with TrueType outlines.
The fonts in this package contain a dummy digital signature
in order to fool Microsoft products into allowing use of additional TrueType features.

\subsection{Source Packages}

You can also download the FontForge sources for all of the fonts
from the \href{https://github.com/typiconman/fonts-cu/}{GitHub repository}.
This is only useful if you are planning on editing the fonts in the
\href{http://fontforge.sourceforge.net}{FontForge} font editor. In general,
you will not gain any productivity improvements from rebuilding the font files,
so rebuilding from source is not recommended unless you have a real need
to modify the fonts, for example, to add your own additional glyphs to the Private Use Area.

\section{System Requirements}

All of these fonts are large Unicode fonts and require a Unicode-aware operating
system and software environment. Outside of a Unicode-aware environment,
you will only be able, at most, to access the first 256 glyphs of a font.

\subsection{Microsoft Windows}

Unicode has been supported since Windows 95, however to use the
OpenType-CFF version of the fonts, you need at least Windows 2000.
You will need a word processor that can handle
Unicode-based documents, such as Microsoft Word 97 and above,
or \href{http://www.libreoffice.org}{LibreOffice}.
Please note that maintenance of OpenOffice.org has been discontinued,
so we recommend using LibreOffice instead. If using \TeX{},
you will need a Unicode-aware \TeX{} engine, such as 
\XeTeX{} or \LuaTeX.

You will also need a way to enter the Unicode characters that are not
directly accessible from standard keyboards. We recommend
installing a Church Slavonic or Russian-Extended keyboard layout, 
avilable from the \href{http://www.ponomar.net/cu_support/keyboard.html}
{Slavonic Computing Initiative website}. It is also possible to enter
characters using the Windows Character Map utility or by codepoint,
but this is not recommended.

\subsection{GNU/Linux}

In order to be able to handle TrueType or OpenType fonts, your system should
have the \href{http://freetype.sourceforge.net}{freetype} library installed
and enabled; this is normally done by default in all modern distributions.
You will need a Unicode-aware word processor, such
as \href{http://www.libreoffice.org}{LibreOffice}.
Please note that OpenOffice.org is no longer maintained,
so we recommend using LibreOffice instead. If using \TeX,
you will need a Unicode-aware \TeX{} engine, such as 
\XeTeX{} or \LuaTeX.

You will need a keyboard driver to input Unicode characters.
Under GNU/Linux, this is handled by the |m17n| library and database. See the \href{http://www.ponomar.net/cu_support/keyboard.html}
{Slavonic Computing Initiative website} for more details.

\subsection{OS X}

Not sure.

\section{Private Use Area}

The Unicode Private Use Area (PUA) is a set of three ranges of codepoints
(U+E000 to U+F8FF, Plane 15 and Plane 16) that are guaranteed to never
be assigned to characters by the Unicode Consortium and can be used by third parties
to allocate their own characters. The Slavonic Computing Initiative has established an
industry standard for character allocation in the PUA, which is described in full
in the \href{http://www.ponomar.net/files/pua_policy.pdf}{PUA Allocation Policy}.

The PUA in these fonts contains various additional glyphs: contextual alternatives,
stylistic alternatives, ligatures, hypothetical and nonce glyphs, various glyphs
not yet encoded in Unicode and various technical symbols. Most of these glyphs
(the alternative glyphs and ligatures) are accessible via 
\hyperlink{OT}{OpenType} and \hyperlink{Graphite}{SIL Graphite} features.
Thus, you generally do not need to access glyphs in the PUA directly. There
may be some exceptions:

\begin{itemize}

\item If you need to access characters not yet encoded in Unicode and nonce glyphs.

\item If you need to access alternative glyphs and ligatures on legacy systems that 
do not support OpenType or Graphite features.

\item If you are a computer programmer and need to work with glyphs
on a low level without relying on OpenType -- having all alternatives
mapped to the PUA allows for a simple way to access glyphs by codepoint
instead of working with glyph indeces, which can change between versions
of a font.

\end{itemize}

For the characters mapped in the PUA and other technical considerations,
please see the \href{http://www.ponomar.net/files/pua_policy.pdf}
{PUA Allocation Policy}.

\section{OpenType Technology}
\hypertarget{OT}{}\label{OT}

OpenType is a “smart font” technology for advanced typography
developed by Microsoft Corporation and Adobe Systems and based on
the TrueType font format. It allows for correct typography in 
complex scripts as well as providing advanced typographic effects.
This is achieved by applying various \textit{features}, or \textit{tags},
described in the OpenType specification. Some of these features are supposed
to be enabled by default, while others are considered optional, and may be
turned on and off by the user when desired.

\subsection{On Microsoft Windows}

In order to use these advanced typographic features,
in addition to a “smart” font (like the fonts in this package), you need
an OpenType-aware application. Not all applications currently support OpenType, 
and not all applications that claim to support OpenType actually support
all features or provide an interface to access features. Older versions of
Microsoft's Uniscribe library did not support OpenType features for
Cyrillic and Glagolitic, but beginning with Windows 7, this has been resolved.

Generally speaking, you will get best results in \XeTeX{} or \LuaTeX{}
using the \texttt{fontspec} package or using advanced desktop publishing software
such as Adobe InDesign. Most OpenType features are also accessible
in Microsoft Office 2010 and later (but see below for details).
LibreOffice also supports OpenType features starting with version 4.1,
however provides no method to turn on optional features. Please
see the section on \hyperlink{Graphite}{SIL Graphite}, below.

\subsection{On GNU/Linux}

OpenType support is provided by the HarfBuzz shaping library, which is 
accessible through FreeType, part of most standard distributions of the X Window
System. Thus, OpenType will be available in any application that uses FreeType,
though many applications lack an interface to turn on and off optional features.
Generally speaking, you will get best results in \XeTeX{} or \LuaTeX{}
using the \texttt{fontspec} package. LibreOffice also supports
OpenType features starting with version 4.1,
however provides no method to turn on optional features. Please
see the section on \hyperlink{Graphite}{SIL Graphite}, below.

\subsection{OpenType Features}

\subsubsection{Combining Mark Positioning}
\hypertarget{mark}{}

OpenType allows smart diacritic positioning: if you type a letter followed by
a diacritic, the diacritic will be placed exactly above or below the letter; this
is provided by the \texttt{mark} feature. In addition, the \texttt{mkmk} feature
is used to position two marks with respect to each other, so that an additional
diacritic can be stacked properly above the first. This behavior is demonstrated
below:

\begin{figure}[h]
\centering
\begin{tabular}{ll}
\large{  {\slv а}  + {\slv ◌́} → {\slv а́ } } &   \\
\large{ {\slv А}  + {\slv ◌́} → {\slv А́ } } & (glyph positioning via \emph{mark} feature) \\
\large{ {\slv ◌ⷭ} + {\slv  ◌‍҇} → {\slv ◌ⷭ҇ } } & (glyph positioning via \emph{mkmk} feature) \\
\end{tabular}
\end{figure}

The fonts provide proper \texttt{mark} and \texttt{mkmk} anchor points
for all Cyrillic and Glagolitic letters and combining marks, allowing you to enter them in
almost any combination (even those that are implausible). Most OpenType renderers
(except older versions of Adobe’s Cooltype library) support these features,
so you should be able to achieve correct positioning in most OpenType-aware
applications (for example, in MS Word 2010 or newer, LibreOffice 4.1 or newer
and \XeTeX{}).

\subsubsection{Glyph Composition and Decomposition}
\hypertarget{ccmp}{}

The Glyph Composition / Decomposition (\texttt{ccmp}) feature is used
to compose two characters into a single glyph for better glyph processing.
This feature is also used to create precomposed forms of a base glyph with
diacritical marks when use of only \texttt{mark} and \texttt{mkmk} cannot achieve
the necessary positioning. It is also used to create alternative glyph shapes,
such as the alternative version of the Psili used over capital letters and
the truncated forms of the letter Uk used with accent marks, as is 
demonstrated in the examples below:

\begin{figure}[h]
\centering
\begin{tabular}{ll}
\large{ {\slv ◌҆} } $\rightarrow$ \large { {\slv  ◌ } } & (glyph substitution using \emph{ccmp} feature) \\
\large{ {\slv ◌҆}  + {\slv ◌̀} $\rightarrow$ {\slv ◌҆̀} } & (ligature substitution using \emph{ccmp} feature) \\
\large{ {\slv т}  + {\slv } + {\slv в} $\rightarrow$ {\slv т‍в } } & (ligature substitution using \emph{ccmp} feature) \\
\large{ {\slv ꙋ}  + {\slv ◌ⷯ} $\rightarrow$ {\slv ꙋⷯ } } & (contextual substitution using \emph{ccmp} feature) \\
\end{tabular}
\end{figure}

Generally speaking, the \emph{ccmp} feature is not supposed to
(and often just cannot) be turned off, and thus this functionality
should work properly in any OpenType-aware application. For more details
on ligatures, see \href{http://www.unicode.org/notes/tn41/}{Unicode
Technical Note \#41: Church Slavonic Typography in Unicode}.

\subsubsection{Language-based Features}

Language-based features such as the \texttt{loca} (localized forms) feature
provide access to language-specific alternate glyph forms, such as the
alternate forms of the Cyrillic Letter I used in Ukrainian and Belarussian:

\begin{figure}[h]
\centering
\begin{tabular}{ll}
\large{  {\slv і } } &  (Church Slavonic text) \\
\large{ {\slv і̇ } } & (Ukrainain text) \\
\end{tabular}
\end{figure}

To make use of these features, you need an OpenType-aware application
that supports specifying the language of text, for example \XeTeX{} or
\LuaTeX{} using the \texttt{fontspec} or \texttt{polyglossia} packages.
Since many software applications do not allow you to specify Church Slavonic
as a language of text, it is assumed by default that the font is being 
used to represent Church Slavonic text, and thus all glyphs have
Church Slavonic appearances unless another language is specified.

LibreOffice allows you to specify that text is in Church Slavonic
starting with version 5.0. This will allow you to take advantage of other
features, such as Church Slavonic hyphenation (see the
\href{http://sci.ponomar.net/}{Slavonic Computing Initiative website}
for more information). Microsoft Corporation does not recognize
Church Slavonic as a valid language, so you will not be able to set 
the language of text to Church Slavonic in any Microsoft
product.\footnote{Please do not contact the font maintainers about this issue.
Instead, complain to Microsoft Customer Service in the USA at 1-800-642-7676 
or in Canada at +1 (877) 568-2495.}

\subsubsection{Stylistic Alternatives and Stylistic Sets}

Stylistic Alternatives (\texttt{salt} feature) provide variant glyph shapes
that may be selected by the user at will. Typically, these are glyphs that differ
from the base glyph only in graphical appearance and the use of these glyphs
does not follow any language-based or typography-based rules, but rather is'
just an embellishment. For example, the following variant forms of U+1F545
Symbol for Marks Chapter are provided:

\begin{center}
\begin{tabular}{ccccc}
U+1F545	& \multicolumn{4}{c}{Alternative Glyphs} \\
\hline
{\slv \Huge 🕅} &	\textcolor{gray}{\slv \Huge } & \textcolor{gray}{\slv \Huge } & \textcolor{gray}{\slv \Huge } & \textcolor{gray}{\slv \Huge }  \\
\hline
\end{tabular}
\end{center}

Stylistic sets are used to enable a group of stylistic variant glyphs,
designed to harmonize visually, and make them automatically substituted
instead of the default forms. OpenType allows to specify up to 20 stylistic
sets, marking them features \texttt{ss01}, \texttt{ss02}\ldots{} \texttt{ss20}. 

Use of Stylistic Alternatives and Stylistic Sets requires an OpenType-aware
application that provides an interface to turn off and on advanced features
(since by default these features are turned off). This is possible in \XeTeX{}
or \LuaTeX{} using the \texttt{fontspec} package and in advanced desktop
publishing software such as Adobe InDesign. Microsoft Office and LibreOffice
currently have no way to access Stylistic Alternatives and Stylistic Sets.
If necessary, you may access these glyphs by codepoint from the Private
Use Area, though this is not recommended.

\section{SIL Graphite Technology}

\hypertarget{Graphite}{}\label{Graphite}

\href{http://scripts.sil.org/Graphite}{Graphite} is a “smart font” technology
developed by \href{http://www.sil.org}{SIL International}. Since, unlike OpenType,
Graphite does not have predefined features, it provides the developer with an ability
to control subtle typographic features that may be difficult or impossible to handle
with OpenType. In fact, Graphite is in some respects more powerful than OpenType,
though this additional power is not necessary for standard Church Slavonic typography.
In addition, while support of OpenType features often varies from application to application,
Graphite relies on a single engine, and thus all Graphite features are supported
whenever an application supports Graphite. However, Graphite is not supported widely:
in addition to SIL's own \href{http://scripts.sil.org/WorldPadDownload}{WorldPad}
editor (a Windows-only application which requires a .NET runtime), Graphite is
supported in LibreOffice (starting with OOo version 3.2), Mozilla Firefox (starting with
version 11), and \XeTeX{} (starting with version 0.997). Graphite support is not
available in Microsoft Office.

Note that it is currently not possible to add Graphite tables to OpenType-CFF fonts.
Therefore Graphite is only supported in the TrueType versions of fonts. Mostly
Graphite will be of interest to LibreOffice users, since LibreOffice does not provide
any interface to turn off and on OpenType features.

\subsection{Graphite in LibreOffice}

LibreOffice (and all OpenOffice.org derivatives starting with version 3.2)
automatically recognizes fonts which contain Graphite tables.
For such fonts Graphite rendering is enabled by default.
Correct positioning, attachment and substitutions will work automatically.
However, there is currently no graphical interface that can be used
to turn additional features on and off. Instead, a special extended font name
syntax has been developed: in order to activate an optional feature, its ID,
followed by an equals sign and the ID of the desired setting, are appended
directly to the font name string. An ampersand is used to separate
different feature/settings pairs.

For example, the following “font” should be used in order to get
alternative glyphs for U+1F545 Symbol for Marks Chapter:

\begin{verbatim}
Ponomar Unicode TT:mark=1
\end{verbatim}

Of course modifying the font name directly is very inconvenient, since
it is difficult to remember short tags and numerical values used for
feature/setting IDs in different fonts. You may try to
install the \href{https://github.com/thanlwinsoft/groooext}
{Graphite Font Extension}, which provides a dialog for easier feature selection.
However, this extension has not been maintained since the passing of its
developer in 2011, and so may not work correctly in later versions of LibreOffice.
If you experience problems with Graphite features, you may get better
results accessing glyphs directly by codepoint from the Private Use Area,
though this is not recommended.

\subsubsection{Using Graphite Features in \XeTeX}

Graphite Support is available in \XeTeX, which means Graphite
features are now accessible from \TeX{} documents. Moreover, it is possible
to enable the Graphite font renderer with the \texttt{fontspec} package,
which greatly simplifies selecting system-installed fonts in \XeTeX{} and \LuaTeX.
This functionality requires at least \TeX{} Live 2010 or Mik\TeX 2.9.

You can activate the Graphite rendering mode for a particular font via the
the \texttt{Renderer} option (its value should be set to \texttt{Graphite})
in the argument list of a font selection command. Since there are no standard
feature tags in Graphite, most of the \texttt{fontspec} feature selection
interface is useless here: optional feature identifiers and their settings are
just passed to the \texttt{RawFeature} option. Please consult the \texttt{fontspec}
documentation for more information.

\section{Ponomar Unicode}

Ponomar Unicode is a typeface that reproduce the typeface of Synodal Church Slavonic
editions from the beginning of the 20th Century. It is intended for working with
modern Church Slavonic texts (Synodal Slavonic).

\subsection{Synodal Church Slavic}

\begin{churchslavonic}
{\slv \large
Бл҃же́нъ мꙋ́жъ, и҆́же не и҆́де на совѣ́тъ нечести́выхъ, и҆ на пꙋтѝ грѣ́шныхъ не ста̀, и҆ на сѣда́лищи гꙋби́телей не сѣ́де: но въ зако́нѣ гдⷭ҇ни во́лѧ є҆гѡ̀, и҆ въ зако́нѣ є҆гѡ̀ поꙋчи́тсѧ де́нь и҆ но́щь. И҆ бꙋ́детъ ꙗ҆́кѡ дре́во насажде́ное при и҆схо́дищихъ во́дъ, є҆́же пло́дъ сво́й да́стъ во вре́мѧ своѐ, и҆ ли́стъ є҆гѡ̀ не ѿпаде́тъ: и҆ всѧ̑, є҆ли̑ка а҆́ще твори́тъ, ᲂу҆спѣ́етъ. Не та́кѡ нечести́вїи, не та́кѡ: но ꙗ҆́кѡ пра́хъ, є҆го́же возмета́етъ вѣ́тръ ѿ лица̀ землѝ. Сегѡ̀ ра́ди не воскре́снꙋтъ нечести́вїи на сꙋ́дъ, нижѐ грѣ̑шницы въ совѣ́тъ првⷣныхъ. Ꙗ҆́кѡ вѣ́сть гдⷭ҇ь пꙋ́ть првⷣныхъ, и҆ пꙋ́ть нечести́выхъ поги́бнетъ.
}
\end{churchslavonic}

\subsection{Kievan Church Slavic}

\begin{churchslavonic}
{\slv \large
Бл҃же́нъ мꙋ́жъ, и҆́же не и҆́ᲁе на совѣ́тъ нечести́выхъ, и҆ на пꙋтѝ грѣ́шныхъ не ста̀, и҆ на сѣᲁа́лищи гꙋби́телей не сѣ́ᲁе: но въ зако́нѣ гᲁⷭ҇ни во́лѧ є҆гѡ̀, и҆ въ зако́нѣ є҆гѡ̀ поꙋчи́тсѧ де́нь и҆ но́щь. И҆ бꙋ́ᲁетъ ꙗ҆́кѡ дре́во насажᲁе́ное при и҆схо́ᲁищихъ во́ᲁъ, є҆́же плоᲁъ сво́й да́стъ во вре́мѧ своѐ, и҆ ли́стъ є҆гѡ̀ не ѿпаᲁе́тъ: и҆ всѧ̑, є҆ли̑ка а҆́ще твори́тъ, ᲂу҆спѣ́етъ. Не та́кѡ нечести́вїи, не та́кѡ: но ꙗ҆́кѡ пра́хъ, є҆го́же воꙁмета́етъ вѣ́тръ ѿ лица̀ землѝ. Сегѡ̀ ра́ᲁи не воскре́снꙋтъ нечести́вїи на сꙋ́ᲁъ, нижѐ грѣ̑шницы въ совѣ́тъ првⷣныхъ. Ꙗ҆́кѡ вѣ́сть гᲁⷭ҇ь пꙋ́ть првⷣныхъ, и҆ пꙋ́ть нечести́выхъ поги́бнетъ.
}
\end{churchslavonic}

\subsection{Other Languages}

The Ponomar Unicode font may also be used to typeset liturgical texts in other languages that use the ecclesiastic Cyrillic alphabet. Three such examples
are fully supported by the font: Romanian (Moldovan) in its Cyrillic alphabet, Aleut (Fox Island or Eastern dialect) in its Cyrillic alphabet, and Yakut (Sakha) as written in the alphabet created by Bishop Dionysius (Khitrov).

Here is an example of the Lord's Prayer in Romanian (Moldovan) Cyrillic: \\

\begin{churchslavonic}
{\slv \large 
Та́тъль но́стрꙋ ка́реле є҆́щй ꙟ҆ Че́рюрй: ᲃ︀фн҃цѣ́скъсе Нꙋ́меле тъ́ꙋ: ві́е ꙟ҆пъръці́ѧ та̀: фі́е во́ѧ та̀, прекꙋ́мь ꙟ҆ Че́рю̆ шѝ пре пъмѫ́нть. Пѫ́йнѣ но́астръ чѣ̀ ᲁепꙋ́рꙋрѣ ᲁъ́не но́аѡ а҆́стъꙁй. Шѝ не ꙗ҆́ртъ но́аѡ греша́леле но́астре, прекꙋ́мь шѝ но́й є҆ртъ́мь греши́цилѡрь но́щри. Ши́ нꙋ́не ᲁꙋ́че пре но́й ꙟ҆ и҆спи́тъ. Чѝ не и҆ꙁбъвѣ́ще ᲁе че́ль ръ́ꙋ. 
} \\
\end{churchslavonic}

And here is an example of the Lord's Prayer in Aleut Cyrillic: \\

\begin{churchslavonic}
{\slv \large
Тꙋмани́нъ А́даԟъ! А҆́манъ акꙋ́х̑тхинъ и́нинъ кꙋ́ҥинъ, А́са́нъ амчꙋг̑а́сѧ́да́г̑та, Аҥали́нъ а҆ԟа́г̑та, Анꙋхтана́тхинъ малга́г̑танъ и́нимъ кꙋ́ганъ ка́юхъ та́намъ кꙋ́ганъ. Ԟалга́дамъ анꙋхтана̀ ҥи̑нъ аԟача́ ꙋ̆а҆ѧ́мъ: ка́юхъ тꙋма́нинъ а́д̑ꙋнъ ҥи̑нъ игни́да, а҆ма́кꙋнъ тꙋ́манъ ка́юхъ малгалиги́нъ ҥи̑нъ ад̑ꙋг̑и́нанъ игнида́кꙋнъ: ка́юхъ тꙋ́манъ сꙋглатачх̑и́г̑анах̑тхинъ, та́г̑а ад̑алю́дамъ илѧ́нъ тꙋ́манъ аг̑г̑ича.
} \\
\end{churchslavonic}

And here is an example of the Lord's Prayer in Yakut (Sakha): \\

\begin{churchslavonic}
{\slv \large
Халланнаръ юрдюлѧригѧрь баръ агабытъ бисенѧ ! Свѧтейдѧннинь а̄тыҥъ эенѧ ; кѧллинь царстваҥъ эенѧ ; сирь юрдюгѧрь кёҥюлюҥь эенѧ , халланъ юрдюгѧрь курдукъ боллунъ ; бюгюҥю кюннѧги асыръ аспытынъ бисенинь кулу бисеха бюгюнь ; бисиги да естѧрбитинь халларъ бисеха , хайтахъ бисиги да халларабытъ беэбить естѧхтѧрбитигѧрь ; килѧримѧ да бисигини альԫархайга ; хата быса бисигини албынтанъ .
}
\end{churchslavonic}

Here is an example using the Typicon symbols from Nikita Syrnikov's book {\slv Клю́чъ къ церко́вномꙋ ᲂу҆ста́вꙋ}:

\begin{churchslavonic}
{\slv \large
і\textcolor{red}{і}꙼̇ ⧟̇҃ іѡа́ннꙋ і̲꙼ на лиⷮ бл҃жеⷩ҇, па́влꙋ пѣⷭ҇ г҃. а і҆оа́ннꙋ ѕ҃.

и᷷͏҃і і\textcolor{red}{і}꙼̇ ⧟̲̇҃ кири́лꙋ і̲꙼ коⷣ и҆ и҆́коⷭ҇ о҆́бщїй и҆ коⷣ а҆фана́сїю.

д҃ 🤉 іі̲ и҆си́дорꙋ ⹇ гео́ргїю кири́лѣ ⹉

}
\end{churchslavonic}

\subsection{Font Features}

This font places some characters in the Private Use Area (PUA).
For the general PUA mappings, please see the
\href{http://www.ponomar.net/files/pua_policy.pdf}{PUA Allocation Policy }.

The font also places some additional characters into the open range
section of the PUA. These are:

\begin{itemize}
\item U+F400 -- Alternatives of SMP glyphs: This section contains copies in the BMP of SMP glyphs for support in legacy applications. Currently, the following are available: U+F400 - U+F405 -- Typicon symbols (copies of U+1F540 through U+1F545)
\item U+F410 -- Presentation forms: Contains various presentation forms and ligatures used internally by the font. Generally, these are not intended to be called by users or external applications.
\item U+F420 -- Linguistic alternatives: Contains alternative shapes of glyphs that are language specific. Right now, these are modern punctuation shapes for use with Latin characters. These are not intended to be called externally.
\item U+F441 and on: stylistic alternatives of Latin characters (Blackletter forms). These can be called via the Stylistic Set 2, but, if necessary, they may be called from the PUA directly. They are mapped to in the same order as in the Basic Latin block, beginning with U+F441 (for U+0041 Latin Capital Letter A). In addition to the Basic Latin repertoire, we also have: U+F4DE -- Blackletter Thorn; U+F4FE -- Lowercase Blackletter Thorn; and U+F575 -- Blackletter Long S
\end{itemize}

The font provides a number of ligatures, which are made by inserting the Zero Width Joiner (U+200D) between two characters. Here is a list of ligatures:

\begin{tabular}{lcc}
Name	& Sequence	& Appearance \\
\hline
Ligature A-U	& U+0430 U+200D U+0443	& {\slv{\large а‍у}}	\\
Ligature El-U	& U+043B U+200D U+0443 & {\slv{\large л‍у}}	\\
Ligature Te-Ve	& U+0442 U+200D U+0432	& {\slv{\large т‍в}}	\\
\hline
\end{tabular}
\\

In OpenType, the following Stylistic alternatives are defined:

\newfontfamily{\salt}[Alternate=0]{Ponomar Unicode}
\newfontfamily{\salta}[Alternate=1]{Ponomar Unicode}
\newfontfamily{\saltb}[Alternate=2]{Ponomar Unicode}
\newfontfamily{\saltc}[Alternate=3]{Ponomar Unicode}
\newfontfamily{\saltd}[Alternate=4]{Ponomar Unicode}
\newfontfamily{\salte}[Alternate=5]{Ponomar Unicode}
\newfontfamily{\saltf}[Alternate=6]{Ponomar Unicode}
\newfontfamily{\saltg}[Alternate=7]{Ponomar Unicode}

\begin{tabular}{lccccccccc}
	& Base Form	& Salt=0	& Salt=1	& Salt=2	& Salt=3 & Salt=4 & Salt=5 & Salt=6 & Salt=7 \\
\hline
U+1F545	& {\slv{\large 🕅 }}	& {\salt\large 🕅} & {\salta\large 🕅} & {\saltb\large 🕅} & {\saltc\large 🕅} & {\saltd\large 🕅} & {\salte\large 🕅} & {\saltf\large 🕅} & {\saltg\large 🕅} \\
\hline
\end{tabular}

For the Cyrillic letters, the stylistic alternatives feature also allows access to truncated letterforms; the order of the alternate forms is always: lower truncation, upper truncation, left truncation, right truncation. The table demonstrates which truncated forms are available:

\begin{tabular}{lccccc}
	& Base Form	& Salt=0	& Salt=1	& Salt=2	& Salt=3  \\
\hline
U+0440	& {\slv{\large р }}	& {\salt\large р} &  \\
U+0443  & {\slv{\large у }}	& {\salt\large у} &  \\
U+0444  & {\slv{\large ф }}	& {\salt\large ф} & {\salta\large ф} \\
U+0445  & {\slv{\large х}}	& {\salt\large х} & {\salta\large х} & {\saltb\large х}  \\
U+0446  & {\slv{\large ц }}	& {\salt\large ц} &  \\
U+0449  & {\slv{\large щ }}	& {\salt\large щ} &  \\
U+0471  & {\slv{\large ѱ }}	& {\salt\large ѱ} &  {\salta\large ѱ}\\
U+A641  & {\slv{\large ꙁ }}	& {\salt\large ꙁ} &  \\
U+A64B  & {\slv{\large ꙋ }}	& {\salt\large ꙋ} & {\salta\large ꙋ} & {\saltb\large ꙋ} \\
\hline
\end{tabular}


There is also defined Stylistic Set 2 (``ss02''), Blackletter forms. When this stylistic set is turned on, 
Latin letters appear in blackletter as opposed to their modern forms. This is useful for setting Latin text side-by-side with Slavonic in some contexts. See the following example:

\newfontfamily{\blackletter}[StylisticSet=2]{Ponomar Unicode}

Regular Text:
{\slv \large The quick brown fox jumps over the lazy dog. 1234567890. А҆ сїѝ словеса̀ по слове́нски. } \\

Blackletter Text:
{\blackletter \large The quick brown fox jumps over the lazy dog. 1234567890. А҆ сїѝ словеса̀ по слове́нски.  } \\

\subsection{SIL Graphite features}

The following features are defined in SIL Graphite:

\newfontfamily{\graph}[Renderer=Graphite]{Ponomar Unicode TT}

The ``Symbol for Mark's Chapter'' (``mark'') feature provides alternatives for the Mark's Chapter Symbol, much like the salt feature in OpenType. 
The following values produce the corresponding results:

\newfontfamily{\graphA}[Renderer=Graphite, RawFeature={Symbol for Mark's Chapter=Alternative 1}]{Ponomar Unicode TT}
\newfontfamily{\graphB}[Renderer=Graphite, RawFeature={Symbol for Mark's Chapter=Alternative 2}]{Ponomar Unicode TT}
\newfontfamily{\graphC}[Renderer=Graphite, RawFeature={Symbol for Mark's Chapter=Alternative 3}]{Ponomar Unicode TT}
\newfontfamily{\graphD}[Renderer=Graphite, RawFeature={Symbol for Mark's Chapter=Alternative 4}]{Ponomar Unicode TT}
\newfontfamily{\graphE}[Renderer=Graphite, RawFeature={Symbol for Mark's Chapter=Alternative 5}]{Ponomar Unicode TT}
\newfontfamily{\graphF}[Renderer=Graphite, RawFeature={Symbol for Mark's Chapter=Alternative 6}]{Ponomar Unicode TT}
\newfontfamily{\graphG}[Renderer=Graphite, RawFeature={Symbol for Mark's Chapter=Alternative 7}]{Ponomar Unicode TT}
\newfontfamily{\graphH}[Renderer=Graphite, RawFeature={Symbol for Mark's Chapter=Alternative 8}]{Ponomar Unicode TT}

\begin{tabular}{lccccc}
	& Base form	& Alternative 1	& Alternative 2	& Alternative 3	& Alternative 4	\\
\hline
U+1F545	& {\graph{\large 🕅 }}	& {\graphA{\large 🕅}} & {\graphB{\large 🕅}} & {\graphC{\large 🕅}} & {\graphD{\large 🕅}}  \\
\hline
& Alternative 5	& Alternative 6	& Alternative 7	& Alternative 8 \\
& {\graphE{\large 🕅}} & {\graphF{\large 🕅}} & {\graphG{\large 🕅}} & {\graphH{\large 🕅}} \\
\hline
\end{tabular}

The ``Truncation'' feature (trnc) provides the same funcionality as stylistic alternatives (for truncation) above. The possible values are: \verb+1+ (lower truncation), \verb+2+ (upper truncation), \verb+3+ (left truncation) and \verb+4+ (right truncation).

The ``Use blackletter characters for Latin'' feature (blck) provides the same functionality as Stylistic Set 2 in OpenType (see above).

% Codecharts could go here, but do we need them?

\section{Fedorovsk Unicode}

% documentation for Fedorovsk will go here

\section{Credits}

The authors would like to thank the following people:

\begin{itemize}

\item Vladislav Dorosh, who allowed his
\href{http://irmologion.ru/fonts.html}{Hirmos} font to be reencoded
in Unicode and modified, leading to the creation of the Ponomar font.

\item Viktor Baranov of the \href{http://www.manuscripts.ru/}{Manuscripts}
project, who allowed the reencoding and modification of his Menaion font.

\item Michael Ivanovich for help in designing the characters for Sakha
(Yakut), partially taken from his Sakha UCS font.

\item Alexey Kryukov, who answered various qestions about FontForge
and whose extensive documentation for the 
\href{https://github.com/akryukov/oldstand/}{Old Standard} font was 
consulted and partially reused.

\end{itemize}


%% PLACE SAMPLE AND CODECHARTS HERE


\end{document}

