
\section{Private Use Area}
This font places some characters in the Private Use Area (PUA). For the general PUA mappings, please see the Ponomar PUA Allocation Policy at \url{http://www.ponomar.net/files/pua_policy.pdf}.

The font also places some additional characters into the ``open range'' section of the PUA. These are:

\begin{itemize}
\item U+F400 -- Alternatives of SMP glyphs: This section contains copies in the BMP of SMP glyphs for support in legacy applications. Currently, the following are available: U+F400 - U+F405 -- Typicon symbols (copies of U+1F540 through U+1F545)
\item U+F410 -- Presentation forms: Contains various presentation forms and ligatures used internally by the font. Generally, these are not intended to be called by users or external applications.
\item U+F420 -- Linguistic alternatives: Contains alternative shapes of glyphs that are language specific. Right now, these are ``modern'' punctuation shapes for use with Latin characters. These are not intended to be called externally.
\item U+F441 and on: stylistic alternatives of Latin characters (Blackletter forms). These can be called via the Stylistic Set 2, but, if necessary, they may be called from the PUA directly. They are mapped to in the same order as in the Basic Latin block, beginning with U+F441 (for U+0041 Latin Capital Letter A). In addition to the Basic Latin repertoire, we also have: U+F4DE -- Blackletter Thorn; U+F4FE -- Lowercase Blackletter Thorn; and U+F575 -- Blackletter Long S
\end{itemize}

\section{OpenType features}

The font provides a number of ligatures, which are made by inserting the Zero Width Joiner (U+200D) between two characters. Here is a list of ligatures:

\begin{tabular}{lcc}
Name	& Sequence	& Appearance \\
\hline
Ligature A-U	& U+0430 U+200D U+0443	& {\glyphfont{\large а‍у}}	\\
Ligature El-U	& U+043B U+200D U+0443 & {\glyphfont{\large л‍у}}	\\
Ligature Te-Ve	& U+0442 U+200D U+0432	& {\glyphfont{\large т‍в}}	\\
\hline
\end{tabular}
\\

In OpenType, the following Stylistic alternatives are defined:
\newfontfamily{\salt}[Alternate=0]{Ponomar Unicode}
\newfontfamily{\salta}[Alternate=1]{Ponomar Unicode}
\newfontfamily{\saltb}[Alternate=2]{Ponomar Unicode}
\newfontfamily{\saltc}[Alternate=3]{Ponomar Unicode}
\newfontfamily{\saltd}[Alternate=4]{Ponomar Unicode}
\newfontfamily{\salte}[Alternate=5]{Ponomar Unicode}
\newfontfamily{\saltf}[Alternate=6]{Ponomar Unicode}
\newfontfamily{\saltg}[Alternate=7]{Ponomar Unicode}

\begin{tabular}{lccccccccc}
	& Base Form	& Salt=0	& Salt=1	& Salt=2	& Salt=3 & Salt=4 & Salt=5 & Salt=6 & Salt=7 \\
\hline
U+1F545	& {\glyphfont{\large 🕅 }}	& {\salt\large 🕅} & {\salta\large 🕅} & {\saltb\large 🕅} & {\saltc\large 🕅} & {\saltd\large 🕅} & {\salte\large 🕅} & {\saltf\large 🕅} & {\saltg\large 🕅} \\
\hline
\end{tabular}

There is also defined Stylistic Set 2 (``ss02''), Blackletter forms. When this stylistic set is turned on, 
Latin letters appear in blackletter as opposed to their modern forms. This is useful for setting Latin text side-by-side with Slavonic in some contexts. See the following example:

\newfontfamily{\blackletter}[StylisticSet=2]{Ponomar Unicode}

Regular Text:
{\glyphfont \large The quick brown fox jumps over the lazy dog. 1234567890. А҆ сїѝ словеса̀ по слове́нски. } \\

Blackletter Text:
{\blackletter \large The quick brown fox jumps over the lazy dog. 1234567890. А҆ сїѝ словеса̀ по слове́нски.  } \\

\section{SIL Graphite features}

The following features are defined in SIL Graphite:

\newfontfamily{\graph}[Renderer=Graphite]{Ponomar Unicode TT}

The ``Symbol for Mark's Chapter'' (``mark'') feature provides alternatives for the Mark's Chapter Symbol, much like the salt feature in OpenType. 
The following values produce the corresponding results:

\newfontfamily{\graphA}[Renderer=Graphite, RawFeature={Symbol for Mark's Chapter=Alternative 1}]{Ponomar Unicode TT}
\newfontfamily{\graphB}[Renderer=Graphite, RawFeature={Symbol for Mark's Chapter=Alternative 2}]{Ponomar Unicode TT}
\newfontfamily{\graphC}[Renderer=Graphite, RawFeature={Symbol for Mark's Chapter=Alternative 3}]{Ponomar Unicode TT}
\newfontfamily{\graphD}[Renderer=Graphite, RawFeature={Symbol for Mark's Chapter=Alternative 4}]{Ponomar Unicode TT}
\newfontfamily{\graphE}[Renderer=Graphite, RawFeature={Symbol for Mark's Chapter=Alternative 5}]{Ponomar Unicode TT}
\newfontfamily{\graphF}[Renderer=Graphite, RawFeature={Symbol for Mark's Chapter=Alternative 6}]{Ponomar Unicode TT}
\newfontfamily{\graphG}[Renderer=Graphite, RawFeature={Symbol for Mark's Chapter=Alternative 7}]{Ponomar Unicode TT}
\newfontfamily{\graphH}[Renderer=Graphite, RawFeature={Symbol for Mark's Chapter=Alternative 8}]{Ponomar Unicode TT}

\begin{tabular}{lccccc}
	& Base form	& Alternative 1	& Alternative 2	& Alternative 3	& Alternative 4	\\
\hline
U+1F545	& {\graph{\large 🕅 }}	& {\graphA{\large 🕅}} & {\graphB{\large 🕅}} & {\graphC{\large 🕅}} & {\graphD{\large 🕅}}  \\
\hline
& Alternative 5	& Alternative 6	& Alternative 7	& Alternative 8 \\
& {\graphE{\large 🕅}} & {\graphF{\large 🕅}} & {\graphG{\large 🕅}} & {\graphH{\large 🕅}} \\
\hline
\end{tabular}



The ``Use blackletter characters for Latin'' feature (blck) provides the same functionality as Stylistic Set 2 in OpenType (see above).

\section{Sample Texts}

\subsection{Synodal Church Slavic}

\begin{russian}
{\glyphfont \large
Бл҃же́нъ мꙋ́жъ, и҆́же не и҆́де на совѣ́тъ нечести́выхъ, и҆ на пꙋтѝ грѣ́шныхъ не ста̀, и҆ на сѣда́лищи гꙋби́телей не сѣ́де: но въ зако́нѣ гдⷭ҇ни во́лѧ є҆гѡ̀, и҆ въ зако́нѣ є҆гѡ̀ поꙋчи́тсѧ де́нь и҆ но́щь. И҆ бꙋ́детъ ꙗ҆́кѡ дре́во насажде́ное при и҆схо́дищихъ во́дъ, є҆́же пло́дъ сво́й да́стъ во вре́мѧ своѐ, и҆ ли́стъ є҆гѡ̀ не ѿпаде́тъ: и҆ всѧ̑, є҆ли̑ка а҆́ще твори́тъ, ᲂу҆спѣ́етъ. Не та́кѡ нечести́вїи, не та́кѡ: но ꙗ҆́кѡ пра́хъ, є҆го́же возмета́етъ вѣ́тръ ѿ лица̀ землѝ. Сегѡ̀ ра́ди не воскре́снꙋтъ нечести́вїи на сꙋ́дъ, нижѐ грѣ̑шницы въ совѣ́тъ првⷣныхъ. Ꙗ҆́кѡ вѣ́сть гдⷭ҇ь пꙋ́ть првⷣныхъ, и҆ пꙋ́ть нечести́выхъ поги́бнетъ.
}
\end{russian}

\subsection{Kievan Church Slavic}

\begin{russian}
{\glyphfont \large
Бл҃же́нъ мꙋ́жъ, и҆́же не и҆́ᲁе на совѣ́тъ нечести́выхъ, и҆ на пꙋтѝ грѣ́шныхъ не ста̀, и҆ на сѣᲁа́лищи гꙋби́телей не сѣ́ᲁе: но въ зако́нѣ гᲁⷭ҇ни во́лѧ є҆гѡ̀, и҆ въ зако́нѣ є҆гѡ̀ поꙋчи́тсѧ де́нь и҆ но́щь. И҆ бꙋ́ᲁетъ ꙗ҆́кѡ дре́во насажᲁе́ное при и҆схо́ᲁищихъ во́ᲁъ, є҆́же плоᲁъ сво́й да́стъ во вре́мѧ своѐ, и҆ ли́стъ є҆гѡ̀ не ѿпаᲁе́тъ: и҆ всѧ̑, є҆ли̑ка а҆́ще твори́тъ, ᲂу҆спѣ́етъ. Не та́кѡ нечести́вїи, не та́кѡ: но ꙗ҆́кѡ пра́хъ, є҆го́же воꙁмета́етъ вѣ́тръ ѿ лица̀ землѝ. Сегѡ̀ ра́ᲁи не воскре́снꙋтъ нечести́вїи на сꙋ́ᲁъ, нижѐ грѣ̑шницы въ совѣ́тъ првⷣныхъ. Ꙗ҆́кѡ вѣ́сть гᲁⷭ҇ь пꙋ́ть првⷣныхъ, и҆ пꙋ́ть нечести́выхъ поги́бнетъ.
}
\end{russian}

\subsection{Other Languages}
The Ponomar Unicode font may also be used to typeset liturgical texts in other languages that use the ecclesiastic Cyrillic alphabet. Three such examples
are fully supported by the font: Romanian (Moldovan) in its Cyrillic alphabet, Aleut (Fox Island or Eastern dialect) in its Cyrillic alphabet, and Yakut (Sakha) as written in the alphabet created by Bishop Dionysius (Khitrov).

Here is an example of the Lord's Prayer in Romanian (Moldovan) Cyrillic: \\

\begin{russian}
{\glyphfont \large 
Та́тъль но́стрꙋ ка́реле є҆́щй ꙟ҆ Че́рюрй: ᲃ︀фн҃цѣ́скъсе Нꙋ́меле тъ́ꙋ: ві́е ꙟ҆пъръці́ѧ та̀: фі́е во́ѧ та̀, прекꙋ́мь ꙟ҆ Че́рю̆ шѝ пре пъмѫ́нть. Пѫ́йнѣ но́астръ чѣ̀ ᲁепꙋ́рꙋрѣ ᲁъ́не но́аѡ а҆́стъꙁй. Шѝ не ꙗ҆́ртъ но́аѡ греша́леле но́астре, прекꙋ́мь шѝ но́й є҆ртъ́мь греши́цилѡрь но́щри. Ши́ нꙋ́не ᲁꙋ́че пре но́й ꙟ҆ и҆спи́тъ. Чѝ не и҆ꙁбъвѣ́ще ᲁе че́ль ръ́ꙋ. 
} \\
\end{russian}

And here is an example of the Lord's Prayer in Aleut Cyrillic: \\

\begin{russian}
{\glyphfont \large
Тꙋмани́нъ А́даԟъ! А҆́манъ акꙋ́х̑тхинъ и́нинъ кꙋ́ҥинъ, А́са́нъ амчꙋг̑а́сѧ́да́г̑та, Аҥали́нъ а҆ԟа́г̑та, Анꙋхтана́тхинъ малга́г̑танъ и́нимъ кꙋ́ганъ ка́юхъ та́намъ кꙋ́ганъ. Ԟалга́дамъ анꙋхтана̀ ҥи̑нъ аԟача́ ꙋ̆а҆ѧ́мъ: ка́юхъ тꙋма́нинъ а́д̑ꙋнъ ҥи̑нъ игни́да, а҆ма́кꙋнъ тꙋ́манъ ка́юхъ малгалиги́нъ ҥи̑нъ ад̑ꙋг̑и́нанъ игнида́кꙋнъ: ка́юхъ тꙋ́манъ сꙋглатачх̑и́г̑анах̑тхинъ, та́г̑а ад̑алю́дамъ илѧ́нъ тꙋ́манъ аг̑г̑ича.
} \\
\end{russian}

And here is an example of the Lord's Prayer in Yakut (Sakha): \\

\begin{russian}
{\glyphfont \large
Халланнаръ юрдюлѧригѧрь баръ агабытъ бисенѧ ! Свѧтейдѧннинь а̄тыҥъ эенѧ ; кѧллинь царстваҥъ эенѧ ; сирь юрдюгѧрь кёҥюлюҥь эенѧ , халланъ юрдюгѧрь курдукъ боллунъ ; бюгюҥю кюннѧги асыръ аспытынъ бисенинь кулу бисеха бюгюнь ; бисиги да естѧрбитинь халларъ бисеха , хайтахъ бисиги да халларабытъ беэбить естѧхтѧрбитигѧрь ; килѧримѧ да бисигини альԫархайга ; хата быса бисигини албынтанъ .
}
\end{russian}
